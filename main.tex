% Copyright 2017 Google Inc.
%
% Licensed under the Apache License, Version 2.0 (the "License");
% you may not use this file except in compliance with the License.
% You may obtain a copy of the License at
%
%     http://www.apache.org/licenses/LICENSE-2.0
%
% Unless required by applicable law or agreed to in writing, software
% distributed under the License is distributed on an "AS IS" BASIS,
% WITHOUT WARRANTIES OR CONDITIONS OF ANY KIND, either express or implied.
% See the License for the specific language governing permissions and
% limitations under the License.
\documentclass[acmsmall]{acmart}
\usepackage{semantic}
\usepackage{array}

\newcolumntype{C}{>{$\displaystyle}c<{$}}
\newcommand{\onecol}[1]{\multicolumn{2}{C}{#1}}

\reservestyle[\mathrel]{\command}{\mathbf}
\command{if[if\;],then[\;then\;],else[\;else\;]}
% space is already provided by ``handle'' for ``cancel''.
\command{handle[\;handle\;],cancel[cancel\;],channel[\;channel]}
\command{sleep[sleep\;],never}
\command{int,string,unit,bool}
\command{datatype[datatype\;],machine[machine\;],fun[fun\;],val[val\;]}
\command{let[let\;],in[\;in\;],end[\;end]}
\command{case[\;case\;],of[\;of\;],and[\;and\;]}
\command{true,false}
\reservestyle{\judge}{\mathit}
\judge{binop[\;binop\;],type[\;type],valid[\;valid]}

\begin{document}

\title[EML]{EML}
\maketitle

\section{Introduction}

EML is a programming language.

To be discussed:
\begin{itemize}
\item Why machines have transitions but datatypes do not.
\item Why events are asynchronous (because events should have a type).
\item Why events are values.
\item Why events are linear. Annoyances with linear type system are
  offset by improvements in operational behavior. For example, event
  cancellation is synchronous in EML, unlike CML where event
  cancellation is subject to garbage collection strategy.
\item Interaction of polymorphism and kinds in evaluation strategy.
\item Tree structure of tasks: a task is composed of subtasks,
  synchronous or asynchronous, that can be cancelled.
\item A synchronous term reflects an evaluation and select loop.  A
  term continues to evaluate other branches when it reaches an event,
  until all events at the leaves have been discovered and can be
  selected upon.
\item Because events are linear, they can be handled locally within
  their fiber.
\end{itemize}

\section{Language Definition}

\subsection{Terms}

Let $i, j, \dots$ denote integer constants, $s$ denote string constants,
$c, c', \dots \in \mathit{DV}$ denote datatype constructors, $C, C',
\dots \in \mathit{MV}$ denote machine constructors, and $x, x', \dots
\in \mathit{PV}$ denote pattern variables.  The term language for EML
is as follows:
\begin{itemize}
\item
  $
  \<binop> \in \mathit{BinOp} ::=
  \mathtt{+} |
  \mathtt{-} |
  \mathtt{*} |
  \mathtt{/} |
  \mathtt{=} |
  \mathtt{>} |
  \mathtt{<} |
  \mathtt{>=} |
  \mathtt{<=} |
  \mathtt{!=}
  $. $\mathtt{+}, \mathtt{-}, \mathtt{*}, \mathtt{/}$ are {\it arithmetic},
  and the remaining operators are {\it comparators}.
\item
  $
  e, e' \in \mathit{Config} ::= i\ |\ s\ |\ e \<binop> e'\ |\ c(e_1, \dots, e_n)
  \ |\ \lambda x : \sigma. e\ |\ C(e_1, \dots, e_n) \ |\ x\ |\ (|e)\ |\ (e; e')
  \ |\ \<if> e \<then> e' \<else> e''\ |\ \<case> e \<of> P_i => e_i
  \ |\ \<let> \<val> x = e \<in> e' \<end>
  \ |\ e!e'\ |\ e?\ |\ e \<handle> \<cancel> \Rightarrow e'\ |\ e : \sigma
  \ |\ \<sleep> i\ |\ \<never>
  $
\item
  $
  P, P' \in \mathit{Pattern} ::= i\ |\ s\ |\ x\ |\ c(e_1, ..., e_n)
  \ |\ C(e_1, ..., e_n)\ |\ |P
  $
\item
  $\sigma \in \mathit{Type} ::= t\ |\ T\ |\ \sigma_1 \to \sigma_2\ |\ \sigma_1 \Rightarrow \sigma_2\ |\ |\sigma$.
\item
  $\Gamma, \Delta, \Lambda \in \mathit{Ctxt} ::= \bullet\ |\ \Gamma, x : \sigma$.
\item
  $D \in \mathit{DDecl} ::= \<datatype> t = c_1 \<of> \sigma_{1,1} * \dots * \sigma_{1,n_1}\ |\ \dots
  \ |\ c_m \<of> \sigma_{m,1} * \dots * \sigma_{m,n_m}$.

  In this case, the uses of $|$ are literal and part of the syntax, rather
  than indicating different constructors.
\item
  $M \in \mathit{MDecl} ::= \<machine> M = c_1 \<of> \sigma_{1,1} * \dots * \sigma_{1,n_1}\ |\ \dots
  \ |\ c_m \<of> \sigma_{m,1} * \dots * \sigma_{m,n_m}
  \ \<and>\
  (c_{j_1} P_{1,1} \dots P_{1,n_1} = e_1)\ \dots\ (c_{j_p} P_{1,1} \dots P_{1,n_p} = e_p)$.

  In this case, the uses of $|$ are literal and part of the syntax, rather
  than indicating different constructors.
\item
  $F \in \mathit{FDecl} ::= \<fun> f P_{1,1} \dots P_{1,n} = e_1\ |\ \dots
  \ |\ f P_{m,1} \dots P_{m,n} = e_m : \sigma$.
\item
  $V \in \mathit{VDecl} ::= \<val> v = e : \sigma$.
\item
  $\Sigma \in \mathit{Sig} ::= \bullet\ |\ \Sigma, D\ |\ \Sigma, M$, i.e. $\Sigma$ is a sequence
  of datatype and machine declarations.
\end{itemize}

Events are sleep, never, channel send and receive, and futures.

The {\it linear} types are types of the form $|\sigma$, $\sigma_1 \to \sigma_2$, and
machines and datatypes that contain any constructor $c_i$ such that
$c_i$ has a parameter type $\sigma_{i,j}$ such that $\sigma_{i,j}$ is
linear. Types that are not linear are referred to as {\it copyable}.

Context variable $\Gamma$ is always used for copyable types. $\Delta, \Lambda$ are
always used for linear types: $\Delta$ is for pattern variables, while
$\Lambda$ is for environment declarations.

We write $\sigma \in \Sigma$ if $\sigma$ is a datatype name $t$ or a machine name $T$
of a declaration in $\Sigma$. Similarly, we write $c \in \Sigma$ if $c$ is a
constructor in a datatype or machine declaration in $\Sigma$.

\subsection{Typing}

\subsubsection{Terms}

Typing for EML has two judgements:
\begin{itemize}
\item $\Gamma; \Delta \vdash_\Sigma M : \sigma$.
\item $\Gamma; \Delta \vdash_\Sigma^c M : \sigma$.
\end{itemize}

\begin{figure}
\centering
\renewcommand{\arraystretch}{2}
\begin{tabular}{CC}
\inference[Var]{x:\sigma \in \Gamma}{\Gamma; \Delta \vdash_\Sigma x : \sigma} &
  \inference[LVar]{x:\sigma \in \Delta}{\Gamma; \Delta \vdash_\Sigma x : \sigma} \\
\inference[Constr]{c:\sigma \in \Sigma}{\Gamma; \Delta \vdash_\Sigma c : \sigma} &
  \inference[LConstr]{}{\Gamma; \Delta \vdash_\Sigma c : \sigma} \\
\inference[Int]{}{\Gamma; \Delta \vdash_\Sigma i : \<int>} &
  \inference[String]{}{\Gamma; \Delta \vdash_\Sigma s : \<string>} \\
\onecol{\inference[Arith]{\Gamma; \Delta \vdash_\Sigma^c e : \<int> & \Gamma; \Delta' \vdash_\Sigma^c e' : \<int>}
                         {\Gamma; \Delta, \Delta' \vdash_\Sigma e \<binop> e' : \<int>}
                         {\mathit {\<binop> arithmetic}}} \\
\onecol{
  \inference[Compare]{\Gamma; \Delta \vdash_\Sigma^c e : \<int> & \Gamma; \Delta' \vdash_\Sigma^c e' : \<int>}
                     {\Gamma; \Delta, \Delta' \vdash_\Sigma e \<binop> e' : \<bool>}
                         {\mathit {\<binop> comparator}}} \\
\onecol{\inference[Seq]{\Gamma; \Delta \vdash_\Sigma^c e : \sigma & \Gamma; \Delta' \vdash_\Sigma^c e' : \tau}
                       {\Gamma; \Delta, \Delta' \vdash_\Sigma (e; e') : \tau}} \\
\onecol{
  \inference[If]{\Gamma; \Delta \vdash_\Sigma^c e : \<bool> & \Gamma; \Delta' \vdash_\Sigma^c e' : \sigma &
                       \Gamma; \Delta' \vdash_\Sigma^c e'' : \sigma}
                {\Gamma; \Delta, \Delta' \vdash_\Sigma \<if> e \<then> e' \<else> e'' : \tau}} \\
\onecol{\inference[Cancel]{\Gamma; \Delta \vdash_\Sigma^c e : \sigma & \Gamma; \Delta' \vdash_\Sigma^c e' : \tau}
                          {\Gamma; \Delta, \Delta' \vdash_\Sigma e \<handle> \<cancel> e' : \sigma}} \\
\inference[Sleep]{\Gamma;\Delta \vdash_\Sigma^c e : \<int>}{\Gamma; \Delta \vdash_\Sigma \<sleep> e : |\<unit>} &
  \inference[Never]{}{\Gamma; \Delta \vdash_\Sigma \<never> : |\sigma} \\
\inference[Send]{\Gamma; \Delta \vdash_\Sigma^c e : \sigma \<channel> & \Gamma; \Delta' \vdash_\Sigma^c e' : |\sigma}
                {\Gamma; \Delta, \Delta' \vdash_\Sigma e!e' : |\<unit>} &
  \inference[Rcv]{\Gamma; \Delta \vdash_\Sigma^c e : \sigma \<channel>}{\Gamma; \Delta \vdash_\Sigma e? : |\sigma} \\
\onecol{\inference[Future]{\Gamma; \Delta \vdash_\Sigma^c e : \sigma}{\Gamma; \Delta \vdash_\Sigma |e : |\sigma}} \\
\inference[L$\lambda$L]{\Gamma, x : \sigma; \Delta \vdash_\Sigma^c e : \tau}{\Gamma; \Delta \vdash_\Sigma \lambda x : \sigma. e : \sigma \to \tau} &
  \inference[L$\lambda$R]{\Gamma; \Delta, x : \sigma \vdash_\Sigma^c e : \tau}{\Gamma; \Delta \vdash_\Sigma \lambda x : \sigma. e : \sigma \to \tau} \\
\inference[$\lambda$L]{\Gamma, x : \sigma; \bullet \vdash_\Sigma^c e : \tau}{\Gamma; \bullet \vdash_\Sigma \lambda x : \sigma. e : \sigma \Rightarrow \tau} &
  \inference[$\lambda$R]{\Gamma; x : \sigma \vdash_\Sigma^c e : \tau}{\Gamma; \bullet \vdash_\Sigma \lambda x : \sigma. e : \sigma \Rightarrow \tau} \\
\onecol{
  \inference[Case]{\Gamma; \Delta \vdash_\Sigma^c e : \sigma & \Gamma, \Gamma_i; \Delta, \Delta_i \vdash_\Sigma^c P_i : \sigma &
                   \Gamma, \Gamma_i; \Delta, \Delta_i \vdash_\Sigma^c e_i : \tau}
                  {\Gamma; \Delta \vdash_\Sigma^c \<case> e \<of> P_i => e_i : \tau}} \\
\inference[Let]{\Gamma; \Delta \vdash_\Sigma^c e : \sigma & \Gamma, x : \sigma; \Delta' \vdash_\Sigma^c e' : \tau}
          {\Gamma; \Delta, \Delta' \vdash_\Sigma \<let> \<val> x = e \<in> e' \<end> : \tau} &
  \inference[LLet]{\Gamma; \Delta \vdash_\Sigma^c e : \sigma & \Gamma; \Delta', x : \sigma \vdash_\Sigma^c e' : \tau}
            {\Gamma; \Delta, \Delta' \vdash_\Sigma \<let> \<val> x = e \<in> e' \<end> : \tau} \\
\onecol{} \\
\onecol{\inference[Empty]{\Gamma; \Delta \vdash_\Sigma e : \sigma}{\Gamma; \Delta \vdash_\Sigma^c e() : \sigma}} \\
\onecol{
  \inference[LApp]{\Gamma; \Delta \vdash_\Sigma e(e'_1, \dots, e'_n) : \sigma \to \tau &
                     \Gamma; \Delta' \vdash_\Sigma^c e'' : \sigma}
                  {\Gamma; \Delta, \Delta' \vdash_\Sigma^c e(e'_1, \dots, e'_n, e'') : \tau}} \\
\onecol{
  \inference[App]{\Gamma; \Delta \vdash_\Sigma e(e'_1, \dots, e'_n) : \sigma \Rightarrow \tau & \Gamma; \Delta' \vdash_\Sigma^c e'' : \sigma}
                 {\Gamma; \Delta, \Delta' \vdash_\Sigma^c e(e'_1, \dots, e'_n, e'') : \tau}}
\end{tabular}
\caption{Typing Rules}\label{fig:typing}
\end{figure}

\subsubsection{Types and Contexts}

The judgements expressing type and context validity are:
\begin{itemize}
\item $\vdash_\Sigma \sigma \<type>$.
\item $\vdash \Sigma \<valid>$.
\end{itemize}

The rules of inference are:
\begin{itemize}
\item $\vdash_\Sigma b \<type>$, for $b$ a base type such as $\<int>$ or
  $\<string>$.
\item $\vdash_\Sigma \<datatype> t = c_1 \<of> \sigma_{1,1} * \dots * \sigma_{1,n_1}\ |\ \dots
  \ |\ c_m \<of> \sigma_{m,1} * \dots * \sigma_{m,n_m} \<type>$ if:
  \begin{itemize}
  \item $t \not\in \Sigma$.
  \item $c_i \not\in \Sigma$ for all $i \in \{1, \dots, m\}$.
  \item $\sigma_{i,j} = t$ or $\vdash_\Sigma \sigma_{i,j_i}\ \<type>$, for all $i \in \{1, \dots, m\}$ and
    $j_i \in \{ 1,\dots,n_i \}$.
  \end{itemize}
\item $\vdash_\Sigma \<machine> M = c_1 \<of> \sigma_{1,1} * \dots * \sigma_{1,n_1}\ |\ \dots
  \ |\ c_m \<of> \sigma_{m,1} * \dots * \sigma_{m,n_m}
  \ \<and>\
  (c_{j_1} P_{1,1} \dots P_{1,n_1} = e_1)\ \dots\ (c_{j_p} P_{1,1} \dots P_{1,n_p} = e_p) \<type>$
  if:
  \begin{itemize}
  \item $T \not\in \Sigma$.
  \item $c_i \not\in \Sigma$ for all $i \in \{1, \dots, m\}$.
  \item $\sigma_{i,j} = t$ or $\vdash_\Sigma \sigma_{i,j_i} \<type>$, for all $i \in \{1, \dots, m\}$ and
    $j_i \in \{ 1,\dots,n_i \}$.
  \item $\Gamma_i; \Delta_i \vdash_{\Sigma, M} P_{i,j} : \sigma_{i,j}$ for appropriate $i$ and
    $j$. Adding $M$ to $\Sigma$ here means adding all of the
    constructors $c_i$ as well. In this case, $\Gamma_i$ and
    $\Delta_i$ should probably be included in the machine declaration,
    although of course they could be inferred.
  \item $\Gamma_i; \Delta_i \vdash_{\Sigma, M} e_i : M$.
  \end{itemize}
\end{itemize}

\subsection{Reduction}

Reduction has several judgements:
\begin{itemize}
  \item $\xi \models e \triangleright_H e'$. Head reduction.
  \item $\xi \models e \triangleright_C^K e'$, for $K \in \{ V, S \}$. Child reduction.
  \item $\xi \models e \triangleright_1 e'$. One-step reduction.
  \item $\xi \models e \Rightarrow e'$. Evaluation.
\end{itemize}

The rules of inference are as follows:
\begin{figure}
\centering
\renewcommand{\arraystretch}{2}
\begin{tabular}{CC}
\inference[BinOp]{i \<binop> j = k}{\xi \models i \<binop> j \triangleright_H k}
  \inference[SeqHead]{}{\xi \models (e; e') \triangleright_H e'} \\
\inference[IfTrue]{}{\xi \models \<if> \<true> \<then> e \<else> e' \triangleright_H e}
  \inference[IfFalse]{}{\xi \models \<if> \<false> \<then> e \<else> e' \triangleright_H e}
\end{tabular}
\begin{tabular}{C}
\inference[KChild]{K = k(e_i) & \xi \models e_i \triangleright_1 e_i'}
  {\xi \models t(e_1 \dots e_i \dots e_n) \triangleright_C^K t(e_1 \dots e_i' \dots e_n)}
\end{tabular}
\begin{tabular}{CC}
\inference[SeqLeft]{\xi \models e_1 \triangleright_1 e_1'}{\xi \models (e_1; e_2) \triangleright_1 (e_1'; e_2)}
  \inference[Seq]{\xi \models e_1 \not\triangleright_1}{\xi \models (e_1; e_2) \triangleright_1 e_2} \\
\onecol{\inference[IfCond]{\xi \models e \triangleright_1 e'''}{\xi \models \<if> e \<then> e' \<else> e'' \triangleright_1
  \<if> e''' \<then> e' \<else> e''}} \\
\onecol{\inference[If]{\xi \models e \not\triangleright_1 & \<if> e \<then> e' \<else> e'' \triangleright_H e'''}
  {\xi \models \<if> e \<then> e' \<else> e'' \triangleright_1 e'''}} \\
\onecol{\inference[Step]{\textit{One of:} &
    \xi \models e \triangleright_C^V e' & \xi \models e \triangleright_H e' & \xi \models e \triangleright_C^S e'}
  {\xi \models e \triangleright_1 e'}}
\end{tabular}
\begin{tabular}{C}
\inference[Trans]{\xi \models e \triangleright_1 e' & \xi \models e' \Rightarrow e''}{\xi \models e \Rightarrow e''}
\end{tabular}
\caption{Reduction Rules}\label{fig:reduction}
\end{figure}

* Didn't put $c!e$, etc. in because those are events as presented here,
but they're synchronous in the interpreter. Need to resolve that
difference.

* Need to add handling of cancellation, at least eventually. Futures
also are not handled currently.

\section{Metatheory}

\subsection{Subject Reduction}

\subsection{Adequacy}

\section{Examples}

\section{Related Work}

\section{Conclusion}

\end{document}
